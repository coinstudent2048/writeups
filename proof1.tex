\documentclass{article}
\usepackage[top=1.0in,bottom=1.0in,left=1.0in,right=1.0in]{geometry}
\usepackage{amsmath,amssymb,amsthm,amsfonts}
\usepackage[utf8]{inputenc}
\usepackage{hyperref}
\usepackage{graphicx}

\newtheorem{definition}{Definition}[section]
\newtheorem{theorem}{Theorem}[section]
\newtheorem{corollary}{Corollary}[theorem]
\newtheorem{lemma}[theorem]{Lemma}
\newtheorem*{remark}{Remark}

\title{Proofs by Reduction \#1}
\author{coinstudent2048}
\date{\today}

\begin{document}

\maketitle

%\begin{abstract}
%\end{abstract}

%\section{The Thing}
\noindent Let Let $\lambda$ be the security parameter. Let $\textsf{Setup}$ be the setup algorithm: $(\mathbb{G}, \mathbb{F})\leftarrow\textsf{Setup}(1^{\lambda})$, where $\mathbb{G}$ is a cyclic group of prime order and $\mathbb{F}$ is its scalar field. Let $Z\in\mathbb{G}$ be the identity element.

The notation $\xleftarrow{\$}$ will be used to denote for a uniformly randomly chosen element, and $(1/x)$ for the modular inverse of $x\in\mathbb{F}$. Lastly, we use additive notation for group operations.

\begin{definition}\label{negl}
A function $f:\mathbb{N}\rightarrow\mathbb{R}$ is \textbf{\em negligible} if for every polynomial $p(\cdot)$ there exists an $N\in\mathbb{N}$ such that for all integers $n>N$ it holds that $f(n)<\frac{1}{p(n)}$.
\end{definition}

\noindent Definition \ref{negl} is copied from Katz \& Lindell. We first prove the following lemma:

\begin{lemma}\label{negl-exp}
If $f:\mathbb{N}\rightarrow\mathbb{R}$ is non-negligible, then $g(n)=f(n)^m$ for any $m\in\mathbb{N}$ and $m>1$ is non-negligible.
\end{lemma}
\begin{proof}
The non-negligibility of $f$ means that there exists a polynomial $p(\cdot)$ such that for all $N\in\mathbb{N}$, there exists an $n>N$ such that $f(n)\ge\frac{1}{p(n)}$. Let $p_f(\cdot)$ be such polynomial and $n_f$ be such $n>N$. Then setting $p_g(\cdot)=p_f(\cdot)^m$ and $n_g=n_f$ suffices for non-negligibility of $g$ because $f(n_f)\ge\frac{1}{p_f(n_f)}\Rightarrow f(n_f)^m\ge\big(\frac{1}{p_f(n_f)}\big)^m$.
\end{proof}

\noindent Lemma \ref{negl-exp} justifies the usage of finite number of breaks in proof by reduction.

\begin{definition}[Discrete Logarithm (DL) Assumption]
DL assumption holds relative to $\emph{\textsf{Setup}}$ if for every $\emph{\textsf{PPT}}$ adversary $\mathcal{A}$, there exists a negligible function $\emph{\textsf{negl}}(\lambda)$ such that
\begin{align*}
\emph{\textsf{Pr}}\left[
\begin{array}{c|c}
    \begin{gathered}
        H = xG
    \end{gathered}
    &
    \begin{gathered}
        (\mathbb{G}, \mathbb{F})\leftarrow\emph{\textsf{Setup}}(1^{\lambda}); G,H\xleftarrow{\$}\mathbb{G}; \\
        x\in\mathbb{F}\leftarrow\mathcal{A}(\mathbb{G}, \mathbb{F}, G, H) \\
    \end{gathered}
\end{array}
\right]
\le \emph{\textsf{negl}}(\lambda).
\end{align*}
\end{definition}

\begin{definition}[``One-time Address'' (OTA) Assumption]
OTA assumption holds relative to $\emph{\textsf{Setup}}$ if for every $\emph{\textsf{PPT}}$ adversary $\mathcal{A}$, there exists a negligible function $\emph{\textsf{negl}}(\lambda)$ such that
\begin{align*}
\emph{\textsf{Pr}}\left[
\begin{array}{c|c}
    \begin{gathered}
        C_1 = k_a U + k_b G \\
        \wedge\ C_2 = (1/k_a)G
    \end{gathered}
    &
    \begin{gathered}
        (\mathbb{G}, \mathbb{F})\leftarrow\emph{\textsf{Setup}}(1^{\lambda}); U,G,C_1,C_2\xleftarrow{\$}\mathbb{G}; \\
        k_a, k_b\in\mathbb{F}\leftarrow\mathcal{A}(\mathbb{G}, \mathbb{F}, U, G, C_1, C_2) \\
    \end{gathered}
\end{array}
\right]
\le \emph{\textsf{negl}}(\lambda).
\end{align*}
\end{definition}

\begin{theorem}
OTA assumption holds if and only if DL assumption holds.
\end{theorem}
\begin{proof}
The proof consists of 2 parts:
\begin{itemize}
    \item \textit{DL is easy $\Rightarrow$ OTA is easy:} Applying the first DL break on $L$ base $G$ will give $1/k_a$, which will trivially give $k_a$. Then applying the second DL break on $K - k_a U$ base $G$ will give $k_b$.
    \item \textit{OTA is easy $\Rightarrow$ DL is easy:} Let $A, B\in\mathbb{G}$ (both not equal to $Z$) be the group elements to find DL for (without loss of generality, $x\in\mathbb{F}$ such that $xA=B$). Then perform the following procedure on $A$:
        \begin{enumerate}
        \item Applying an OTA break on $(U, A)$ will give $k_a, k_b\in\mathbb{F}$ such that $U = k_a U + k_b G$ and $A = (1/k_a)G$. Hence, $k_a A = G \Rightarrow U = k_a U + k_b (k_a A)$.
        \item Let $y_1\in\mathbb{F}$ such that $U=y_1 A$. Now $U = k_a U + k_b k_a A$ becomes $y_1 A = k_a (y_1 A) + k_b k_a A \Rightarrow y_1 = k_a y_1 + k_b k_a \Rightarrow y_1 - k_a y_1 = k_b k_a$. Therefore,
        \begin{align*}
            y_1 = k_b k_a (1/(1 - k_a)).
        \end{align*}
        \end{enumerate}
Then perform the same procedure on $B$ (hence another OTA break). Now we have $y_1, y_2\in\mathbb{F}$ such that $U=y_1 A$ and $U=y_2 B$. Hence, $y_1 A = y_2 B \Rightarrow  y_1(1/y_2) A = B$.
\end{itemize}
This completes the proof.
\end{proof}

\begin{definition}[DL ``Vector'' Assumption]\label{vector}
DL Vector assumption holds relative to $\emph{\textsf{Setup}}$ if for all $n>1$ and for every $\emph{\textsf{PPT}}$ adversary $\mathcal{A}$, there exists a negligible function $\emph{\textsf{negl}}(\lambda)$ such that
\begin{align*}
\emph{\textsf{Pr}}\left[
\begin{array}{c|c}
    \begin{gathered}
        \exists z_i(z_i \neq 0), i\in\{1, \ldots, n\} \\
        \wedge\ \sum_{i=1}^{n}{z_i G_i}=H
    \end{gathered}
    &
    \begin{gathered}
        (\mathbb{G}, \mathbb{F})\leftarrow\emph{\textsf{Setup}}(1^{\lambda}); \\
        G_1,\ldots, G_n, H\xleftarrow{\$}\mathbb{G}; \\
        z_1, \ldots, z_n\in\mathbb{F}\leftarrow\mathcal{A}(\mathbb{G}, \mathbb{F}, G_1, \ldots, G_n, H) \\
    \end{gathered}
\end{array}
\right]
\le \emph{\textsf{negl}}(\lambda).
\end{align*}
\end{definition}

\begin{theorem}
DL Vector assumption holds if and only if DL assumption holds.
\end{theorem}
\begin{proof}
The proof consists of 2 parts:
\begin{itemize}
    \item \textit{DL is easy $\Rightarrow$ DL Vector is easy:} Set random scalars on $z_2,\ldots,z_n$ so that at least one of them is not zero. Then applying DL break on $H - \sum_{i=2}^{n}{z_i G_i}$ base $G_1$ will give $z_1$.
    \item \textit{DL Vector is easy $\Rightarrow$ DL is easy:} Assume that there exists an $n>1$ such that finding $z_1,\ldots,z_n\in\mathbb{F}$ satisfying the properties in Definition \ref{vector} is ``easy''. Let $A, B\in\mathbb{G}$ (both not equal to $Z$) be the group elements to find DL for (without loss of generality, $x\in\mathbb{F}$ such that $xA=B$). Then applying a DL Vector break on $G_1=A, G_2=B, G_3=\ldots=G_n=A, H=Z$ will give $z_1, z_2$ such that $z_1 A + z_2 B = Z$. Now we have $z_1 A + z_2 (xA) = Z \Rightarrow z_1 + z_2 x = 0 \Rightarrow x = (-z_1)(1/z_2)$.

Note that both $z_1$ and $z_2$ will never be zero, because if one of them is, then the other should also be zero, contradicting the requirement $\exists z_i(z_i \neq 0)$.
\end{itemize}
This completes the proof.
\end{proof}

%\bibliographystyle{plain}
%\bibliography{main}

\end{document}
