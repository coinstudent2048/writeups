\documentclass{article}
\usepackage[top=1.0in,bottom=1.0in,left=1.0in,right=1.0in]{geometry}
\usepackage{amsmath,amssymb,amsthm,amsfonts}
\usepackage[utf8]{inputenc}
\usepackage{hyperref}
\usepackage{graphicx}

\newtheorem{definition}{Definition}[section]
\newtheorem{theorem}{Theorem}[section]
\newtheorem{corollary}{Corollary}[theorem]
\newtheorem{lemma}[theorem]{Lemma}
\newtheorem*{remark}{Remark}

\title{Proofs by Reduction \#1}
\author{coinstudent2048}
\date{\today}

\begin{document}

\maketitle

%\begin{abstract}
%\end{abstract}

\section{Mathematics}
Let $\mathbb{N}=\{0,1,2,\ldots\}$ and let $\mathbb{R}_{\ge 0}=\{x\in\mathbb{R}\mid x\ge 0\}$. Let $\lambda$ be the security parameter. Let $\textsf{Setup}$ be the setup algorithm: $(\mathbb{G}, \mathbb{F})\leftarrow\textsf{Setup}(1^{\lambda})$, where $\mathbb{G}$ is a cyclic group of prime order and $\mathbb{F}$ is its scalar field. Let $Z\in\mathbb{G}$ be the identity element.

The notation $\xleftarrow{\$}$ will be used to denote for a uniformly randomly chosen element, and $(1/x)$ for the modular inverse of $x\in\mathbb{F}$. Lastly, we use additive notation for group operations.

\begin{definition}\label{negl1}
A function $f:\mathbb{N}\rightarrow\mathbb{R}_{\ge 0}$ is \textbf{\em negligible} if for all positive polynomial $p(\cdot)$ there exists an $N\in\mathbb{N}$ such that for all integers $n>N$ it holds that $f(n)<\frac{1}{p(n)}$.
\end{definition}

\begin{definition}\label{negl2}
An equivalent formulation of negligibility for $f:\mathbb{N}\rightarrow\mathbb{R}_{\ge 0}$ is if for all $c\in\mathbb{N}$ there exists an $N\in\mathbb{N}$ such that for all integers $n>N$ it holds that $f(n)<n^{-c}$.
\end{definition}

\noindent Definitions \ref{negl1} and \ref{negl2} is copied from Katz \& Lindell. We first prove the following lemma:

\begin{lemma}\label{negl-exp}
If $f:\mathbb{N}\rightarrow\mathbb{R}_{\ge 0}$ is non-negligible, then $g(n)=f(n)^m$ for any $m\in\mathbb{N}$ and $m>1$ is non-negligible.
\end{lemma}
\begin{proof}
We use the first definition. Non-negligibility of $f$ means that there exists a positive polynomial $p(\cdot)$ such that for all $N\in\mathbb{N}$, there exists an $n>N$ such that $f(n)\ge\frac{1}{p(n)}$. Let $p_f(\cdot)$ be such polynomial and $n_f$ be such $n>N$. Then setting $p_g(\cdot)=p_f(\cdot)^m$ and $n_g=n_f$ suffices for non-negligibility of $g$ because $f(n_f)\ge\frac{1}{p_f(n_f)}\implies f(n_f)^m\ge\big(\frac{1}{p_f(n_f)}\big)^m$.
\end{proof}
\noindent Lemma \ref{negl-exp} justifies the usage of finite number of breaks of one hardness assumption in proofs by reduction.
\\\\
\noindent I'll credit the proof idea of the following lemma to Atomfried (\texttt{@atomfried:matrix.org}).

\begin{lemma}\label{negl-prod}
If there exists an $n'>N$ that makes both $f_1:\mathbb{N}\rightarrow\mathbb{R}_{\ge 0}$ and $f_2:\mathbb{N}\rightarrow\mathbb{R}_{\ge 0}$ non-negligible, then $g(n)=f_1(n)f_2(n)$ is non-negligible.
\end{lemma}
\begin{proof}
We use the second definition. Non-negligibility of $f$ means that there exists a $c\in\mathbb{N}$ such that for all $N\in\mathbb{N}$, there exists an $n>N$ such that $f(n)\ge n^{-c}$. Let $c_1, c_2$ be such $c\in\mathbb{N}$ for $f_1$ and $f_2$ respectively, and let $n'$ be, as being said in the lemma statement, such $n>N$ for both $f_1$ and $f_2$. Then setting $c_g=c_1 + c_2$ and $n_g=n'$ suffices for non-negligibility of $g$ because $f_1(n')\ge (n')^{-c_1} \wedge f_2(n')\ge (n')^{-c_2} \implies f_1(n')f_2(n') \ge (n')^{-c_1}  (n')^{-c_2} = (n')^{-(c_1+c_2)}$.
\end{proof}

\noindent Lemma \ref{negl-prod} can be used to further generalize Lemma \ref{negl-exp} to products of functions $g(n)=\prod_{i=1}^m f_i(n)$ through induction, \textit{as long as} there exists an $n'>N$ that makes all $f_i$ non-negligible.

\begin{definition}[Discrete Logarithm (DL) Assumption]
DL assumption holds relative to $\emph{\textsf{Setup}}$ if for all $\emph{\textsf{PPT}}$ adversary $\mathcal{A}$, there exists a negligible function $\emph{\textsf{negl}}(\lambda)$ such that
\begin{align*}
\emph{\textsf{Pr}}\left[
\begin{array}{c|c}
    \begin{gathered}
        H = xG
    \end{gathered}
    &
    \begin{gathered}
        (\mathbb{G}, \mathbb{F})\leftarrow\emph{\textsf{Setup}}(1^{\lambda}); G,H\xleftarrow{\$}\mathbb{G}; \\
        x\in\mathbb{F}\leftarrow\mathcal{A}(\mathbb{G}, \mathbb{F}, G, H) \\
    \end{gathered}
\end{array}
\right]
\le \emph{\textsf{negl}}(\lambda).
\end{align*}
\end{definition}

\begin{definition}[``Linking Tag'' (LT) Assumption]
LT assumption holds relative to $\emph{\textsf{Setup}}$ if for all $\emph{\textsf{PPT}}$ adversary $\mathcal{A}$, there exists a negligible function $\emph{\textsf{negl}}(\lambda)$ such that
\begin{align*}
\emph{\textsf{Pr}}\left[
\begin{array}{c|c}
    \begin{gathered}
        C_1 = t_k G + k_a X + k_b U \\
        \wedge\ C_2 = (k_b/k_a)U
    \end{gathered}
    &
    \begin{gathered}
        (\mathbb{G}, \mathbb{F})\leftarrow\emph{\textsf{Setup}}(1^{\lambda}); G,X,U,C_1,C_2\xleftarrow{\$}\mathbb{G}; \\
        t_k,k_a,k_b\in\mathbb{F}\leftarrow\mathcal{A}(\mathbb{G}, \mathbb{F}, G, X, U, C_1, C_2) \\
    \end{gathered}
\end{array}
\right]
\le \emph{\textsf{negl}}(\lambda).
\end{align*}
\end{definition}

\begin{theorem}
LT assumption holds if and only if DL assumption holds.
\end{theorem}
\begin{proof}
The proof consists of 2 parts:
\begin{itemize}
\item \textit{DL is broken $\implies$ LT is broken:} Assume that $\mathcal{A}$ can break DL with non-negligible probability. Applying the first DL break on $C_2$ base $U$ will give $k_b/k_a$.  $\mathcal{A}$ sets a random $k_a$ and computes $k_b = k_a(k_b/k_a)$. Then applying the second DL break on $C_1 - k_a X - k_b U$ base $G$ will give $t_k$.

\item \textit{LT is broken $\implies$ DL is broken:} Assume that $\mathcal{A}$ can break LT with non-negligible probability. Let $A, B\in\mathbb{G}$ (both not equal to $Z$) be the group elements to find DL for (without loss of generality, $x\in\mathbb{F}$ such that $B=xA$). Then perform this procedure for $A$: applying an LT break on $(C_1, C_2) = (G, A)$ will give $t_k, k_a, k_b\in\mathbb{F}$ such that $G = t_k G + k_a X + k_b U$ and $A = (k_b/k_a)U$. Note that both $k_a$ and $k_b$ will never be $0$ because if one of them is, then $A \ne (k_b/k_a)U$, a contradiction.

Then perform the same procedure for $B$ (hence another LT break). Now we have $y_1, y_2\in\mathbb{F}$ such that $A=y_1 U$ and $B=y_2 U$. Hence $B = y_2(1/y_1)A$.
\end{itemize}
This completes the proof.
\end{proof}

%\bibliographystyle{plain}
%\bibliography{main}

\end{document}