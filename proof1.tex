\documentclass{article}
\usepackage[top=1.0in,bottom=1.0in,left=1.0in,right=1.0in]{geometry}
\usepackage{amsmath,amssymb,amsthm}
\usepackage[utf8]{inputenc}
\usepackage{hyperref}

\newtheorem{definition}{Definition}[section]
\newtheorem{theorem}{Theorem}[section]
\newtheorem{corollary}{Corollary}[theorem]
\newtheorem{lemma}[theorem]{Lemma}

\title{Insert Your Awesome Title Here}
\author{coinstudent2048}
\date{\today}

\begin{document}

\maketitle

%\begin{abstract}
%\end{abstract}

\section{Main Thing}
\noindent Let $\mathbb{G}$ be a cyclic group of prime order $l>3$ and $\mathbb{F}$ be its scalar field.

\begin{definition}[Discrete Logarithm (DL) Assumption]
Let $G, H\in\mathbb{G}$. Then finding (the unique) $x\in\mathbb{F}$ such that $xG=H$ is ``hard''.
\end{definition}

\begin{definition}[``One-time Address'' (OTA) Assumption]
Let $U, G\in\mathbb{G}$ whose DL relationship to each other is unknown. Let $f:\mathbb{F}\times\mathbb{F}\rightarrow\mathbb{G}\times\mathbb{G}$ be the following:
\begin{align*}
    (k_a, k_b) \mapsto (k_a U + k_b G, (1/k_a)G)
\end{align*}
Then given $(K, L)\in\mathbb{G}\times\mathbb{G}$, finding (the unique) $f^{-1}(K,L)$ is ``hard''.
\end{definition}

\begin{theorem}
DL assumption is hard if and only if OTA assumption is hard.
\end{theorem}
(This may be false though, see ``Comment'')
\begin{proof}
The proof consists of 2 parts:
    \begin{itemize}
    \item \textit{DL is easy $\Rightarrow$ OTA is easy:} Applying the first DL break on $log_G(L)$ will give $1/k_a$, which will trivially give $k_a$. Then applying the second DL break on $log_G(K - k_a U)$ will give $k_b$.
    \item \textit{OTA is easy $\Rightarrow$ DL is easy:} Let $A, B\in\mathbb{G}$ be the group elements to find DL for (without loss of generality, $x\in\mathbb{F}$ such that $xA=B$). Then perform the following procedure on $A$:
        \begin{enumerate}
        \item Applying an OTA break on $(U, A)$ will give $k_a, k_b\in\mathbb{F}$ such that $U = k_a U + k_b G$ and $A = (1/k_a)G$. Hence, $k_a A = G \Rightarrow U = k_a U + k_b (k_a A)$.
        \item Let $y_1\in\mathbb{F}$ such that $U=y_1 A$. Now $U = k_a U + k_b k_a A$ becomes $y_1 A = k_a (y_1 A) + k_b k_a A \Rightarrow y_1 = k_a y_1 + k_b k_a \Rightarrow y_1 - k_a y_1 = k_b k_a$. Therefore,
        \begin{align*}
            y_1 = k_b k_a (1/(1 - k_a)).
        \end{align*}
        \end{enumerate}
Then perform the same procedure to $B$. Now we have $y_1, y_2\in\mathbb{F}$ such that $U=y_1 A$ and $U=y_2 B$. Hence, $y_1 A = y_2 B \Rightarrow  y_1(1/y_2) A = B$.
    \end{itemize}
\end{proof}
\section{Comment}
\noindent I have doubt on the whole proof, because of the usage of two DL breaks and two OTA breaks. Upon briefly looking on the formal version of ``easy'' and ``hard'' (negligible function), I cannot connect my intuition to it yet.

%\bibliographystyle{plain}
%\bibliography{main}
\end{document}
