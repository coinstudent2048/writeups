\documentclass{article}
\usepackage[top=1.0in,bottom=1.0in,left=1.0in,right=1.0in]{geometry}
\usepackage{amsmath,amssymb,amsthm,amsfonts}
\usepackage[utf8]{inputenc}
\usepackage{hyperref}
\usepackage{mathtools}
%\usepackage{graphicx}

\newtheorem{definition}{Definition}[section]
\newtheorem{theorem}{Theorem}[section]
\newtheorem{corollary}{Corollary}[theorem]
\newtheorem{lemma}[theorem]{Lemma}
\newtheorem*{remark}{Remark}

\title{A Report on Seraphis}
\author{coinstudent2048}
\date{\today}

\begin{document}

\maketitle

\begin{abstract}
This document contains a concise description of Seraphis \cite{seraphis}, a novel privacy-preserving transaction protocol abstraction, and its security model. This document will also serve as a suggestion to the organization of the contents of the final Seraphis paper.
\end{abstract}

\section{Introduction}
In a p2p (peer-to-peer) electronic cash system, the entire supply of currency exists as a digital record that can be stored by any person, and transactions (attempts to transfer money to new owners) are mediated by a network of \textit{peers} (usually called \textit{nodes})... ...An unfortunate consequence of cryptocurrencies being decentralized is that the ledger is \textit{public}, implying that all e-notes and transaction events are public knowledge. If amounts are in cleartext, addresses are trivially traceable, and e-notes to be spent are referenced directly, then observers can discern many details about users’ finances.

Hence, several confidential transaction protocols have been proposed to ensure financial privacy and currency fungibility. [List privacy technologies and limitations]
\subsection{Our contribution}
We introduce Seraphis, privacy-preserving transaction protocol abstraction, which means the unlike previous... Moreover...
\subsection{Acknowledgments}
Lelantus Spark

\section{Preliminaries}
\subsection{Public parameters}
Let $\lambda$ be the security parameter. Let $\mathbb{G}$ be a prime order group based on $\lambda$ where the Discrete Logarithm (DL) and Decisional Diffie-Hellman (DDH) problems are hard, and let $\mathbb{F}$ be its scalar field. Let $G, H, X, U$ be generators of $\mathbb{G}$ with unknown DL relationship to each other. Note that these generators may be produced using public randomness. Let $a_{max}\in\mathbb{F}$ (to be used in range proofs) and $s\in\mathbb{N}$ (to be used in membership proofs). Let $\mathcal{H}:\{0,1\}^*\rightarrow\mathbb{F}$ be a cryptographic hash function. We work in the random oracle model: assume that $\mathcal{H}$ is such. We add a subscript to $\mathcal{H}$, such as $\mathcal{H}_1$, in lieu of domain-separating the hash function explicitly; any domain-separation method may be used in practice. All these public parameters are collected as $pp$, and we now define the setup algorithm: $pp\leftarrow\textsf{Setup}(1^{\lambda})$. $\textsf{Setup}$ is implicitly executed by all players involved in the beginning, hence it can be omitted in protocol descriptions.

The notation $\xleftarrow{\$}$ will be used to denote for a uniformly randomly chosen element, and $(1/x)$ for the modular inverse of $x\in\mathbb{F}$. Lastly, we use additive notation for group operations.

\subsection{E-notes and e-note images}
\begin{definition}\label{e-note}
An \textbf{\em e-note} for scalars $k_a^o, k_b^o, a \in\mathbb{F}$ is a tuple $(C, K^o, m)$ such that $C = x G + a H$ for $x\xleftarrow{\$}\mathbb{F}$, $K^o=k_a^o X + k_b^o U$, and $m$ is an arbitrary data.
\end{definition}
$C$ is called the \textbf{amount commitment} for the amount $a$ with blinding factor $x$, $K^o$ is called the \textbf{one-time address} for (one-time) private key $k_a^o$ and $k_b^o$ (the $o$ superscript indicates ``one-time''), and $m$ is the \textbf{memo field}. We say that someone \textit{owns} an e-note if they know the corresponding scalars $k_a^o, k_b^o, a, x \in\mathbb{F}$.

\begin{definition}\label{e-note-img}
An \textbf{\em e-note image} for an e-note $(C, K^o, m)$ is a tuple $(C', K'^o, \tilde{K})$ such that
\begin{align*}
C' &= t_c G + C \\ &= (t_c+x)G + aH \\ &= v_c G + aH \ , \\
K'^o &= t_k G + K^o \\ &= t_k G + k_a^o X + k_b^o U \ ,\ \text{and} \\
\tilde{K} &= (k_b^o/k_a^o)U
\end{align*}
for $t_c, t_k \xleftarrow{\$}\mathbb{F}$ and independent to each other.
\end{definition}
$C'$ is called the \textbf{masked amount commitment}, $K'^o$ is called the \textbf{masked address}, and $\tilde{K}$ is called the \textbf{linking tag}.

\begin{definition}\label{recv-addr}
A \textbf{\em receiver address} is a tuple $(K^{dh}, K^v, K^s)$  such that $K^{dh}\in\mathbb{G}$, $K^v = k^v K^{dh}$, and $K^s = k_a^s X + k_b^s U$.
\end{definition}
$K^{dh}$ is called the \textbf{Diffie-Hellman base public key}, the $v$ superscript indicates ``view'', and the $s$ superscript indicates ``spend''. The reason for the name of $K^{dh}$ will be clear in the next subsection, while the reason for the names of superscripts will be discussed in the addressing schemes (Subsection \ref{addr-scheme}). We say that someone \textit{owns} a receiver address if they know the corresponding scalars $k^v, k_a^s, k_b^s \in\mathbb{F}$.

\subsection{Authenticated symmetric encryption scheme}
We require the use of an authenticated symmetric encryption scheme. The Diffie-Hellman base public key enables shared secrets between the sender and the receiver, which can be used to produce the key for encryption and the authentication tag. We denote the encryption and decryption of data $x$ with the input $k$ for Key Derivation Function (KDF) as ${\tt enc}[k](x)$ and ${\tt dec}[k](x)$, respectively. We put overlines (e.g. $\overline{x}$) to indicate encrypted data.

The required security properties for application to Seraphis are described in Section \ref{sec-symm}.

\section{A Seraphis transaction}\label{ser-tx}
We now describe a simple Seraphis transaction. This will be used as the basis for further instantiations and modifications (Section \ref{inst}) and for the security model (Section \ref{sec}).

Suppose that Alice would send $a_t\in\mathbb{F}$ amount of funds to Bob. Alice owns a set of e-notes $\{(C_i,K_i^o,m_i)\}_{i=1}^n$ with a total amount of $\big(\sum_{i=1}^{n}{a_i}\big)\ge a_t$, all \textit{connected} to a receiver address $(K_{ali}^{dh}, K_{ali}^v, K_{ali}^s)$. This ``connection'' will be elaborated later on. On the other hand, Bob owns a receiver address $(K_{bob}^{dh}, K_{bob}^v, K_{bob}^s)$. For Bob to receive the funds, he will now send his receiver address to Alice. Alice will actually send funds to two addresses: to Bob's and to herself (for the ``change'' $a_{c} = \sum_{i=1}^{n}{a_i} - a_t$ \textit{even if} $a_{c}=0$). Hence, Alice must create 2 new e-notes. She starts the transaction by doing the following:
\begin{enumerate}
    \item Generate $r_{ali}, r_{bob}\xleftarrow{\$}\mathbb{F}$ and independent to each other.
    \item Compute $R_{ali} = r_{ali}K_{ali}^{dh}$ and $R_{bob} = r_{bob}K_{bob}^{dh}$, then store $R_{ali}$ and $R_{bob}$ to new (empty) memos $m_{ali}$ and $m_{bob}$, respectively. The name for $K^{dh}$ should now be clear.
    \item Compute the sender-receiver shared secrets $q_{ali} = \mathcal{H}_1(r_{ali}K_{ali}^{v})$ and $q_{bob} = \mathcal{H}_1(r_{bob}K_{bob}^{v})$.
    \item Compute the one-time addresses $K_{ali}^o = \mathcal{H}_2(q_{ali})X + K_{ali}^s$ and $K_{bob}^o = \mathcal{H}_2(q_{bob})X + K_{bob}^s$. The $\mathcal{H}_2(q_{ali})$ and $\mathcal{H}_2(q_{bob})$ are uniformly random because of $r_{ali}, r_{bob}$, and random oracle $\mathcal{H}$.
    \item Compute the amount commitments $C_{ali} = \mathcal{H}_3(q_{ali})G + a_c H$ and $C_{bob} = \mathcal{H}_3(q_{bob})G + a_t H$. The blinding factors $\mathcal{H}_3(q_{ali})$ and $\mathcal{H}_3(q_{bob})$ are uniformly random because of $r_{ali}, r_{bob}$, and random oracle $\mathcal{H}$.
    \item Encrypt the amounts: $\overline{a_c} = {\tt enc}[q_{ali}](a_c)$ and $\overline{a_t} = {\tt enc}[q_{bob}](a_t)$, and store $\overline{a_c}$ and $\overline{a_t}$ to memos $m_{ali}$ and $m_{bob}$, respectively.
\end{enumerate}
Alice now has two new e-notes: ${\tt enote}_{ali} = (C_{ali}, K_{ali}^o, m_{ali})$ and ${\tt enote}_{bob} = (C_{bob}, K_{bob}^o, m_{bob})$. These will then be stored to a new (empty) \textit{whole transaction} $T$. Other objects that will be stored to the whole transaction are from proving systems, which can be executed in \textit{any} order. Proving systems are discussed in the next subsections.

For specific instances of Seraphis, there might be changes in some parts of the above steps, and by reflection, in some parts of the Receipt. Here are some notable changes:
\begin{itemize}
\item For some addressing schemes (Subsection \ref{addr-scheme}), the input to $\mathcal{H}_2$, the input to $\mathcal{H}_3$, and the key for both ${\tt enc}$ and ${\tt dec}$ may be constructed differently and different to each other. Nevertheless, these inputs and key must be random sender-receiver shared secrets.
\item A Seraphis transaction can easily have multiple receivers aside from Bob, which implies that Alice will create more than 2 new e-notes. See Subsection \ref{sec-disc} for a discussion of how this affects the security model.
\item A Seraphis transaction can be collaboratively constructed by multiple players. This is the subject of the so-called ``proof dependency'' (Subsection \ref{proof-dep}). Also see Subsection \ref{sec-disc} for a discussion of how this affects the security model.
\end{itemize}

\subsection{Ownership and unspentness proofs}\label{own-unsp}
For each of Alice's owned e-notes $\{(C_i,K_i^o,m_i)\}_{i=1}^n$, Alice must do the following:
\begin{enumerate}
    \item If the masked address $K_i'^o$ is already in the e-note image ${\tt enimg}_i$ in $T$, then go to next step. Else generate $K_i'^o$ from $(C_i, K_i^o, m_i)$ as per definition, and insert it to ${\tt enimg}_i$ in $T$.
    \item If the linking tag $\tilde{K}_i$ is already in ${\tt enimg}_i$ in $T$, then go to next step. Else generate $\tilde{K}_i$ from $(C_i, K_i^o, m_i)$ as per definition, and insert it to ${\tt enimg}_i$ in $T$.
    \item Prepare the proof transcript $\Pi_{\text{o\&u}, i}$ for a non-interactive proving system for the following relation:
$$\{(G, X, U, K_i'^o, \tilde{K}_i\in\mathbb{G}; t_{k,i}, k_{a,i}^o, k_{b,i}^o\in\mathbb{F}): k_{a,i}^o \ne 0 \wedge K_i'^o = t_{k,i} G + k_{a,i}^o X + k_{b,i}^o U \wedge \tilde{K}_i = (k_{b,i}^o/k_{a,i}^o)U \}$$
    \item Append $\Pi_{\text{o\&u}, i}$ to $({\tt enimg}_i, \ldots)$ in $T$.
\end{enumerate}
Aside from verifying the proof transcript, the Verifier must confirm that the linking tags do not yet appear in the ledger.

The required security properties for application to Seraphis are described in Section \ref{prov-prop}. Unlike what's presented above, we recommend a composition proving system in which instead of one $\Pi_{\text{o\&u}, i}$ per $i$, Alice only needs to produce one proof transcript for all $i$'s. We present this in Appendix \ref{comp-prov}, and provide proof that it satisfies the required security requirements.

\subsection{Amount balance}\label{amt-bal}
For each of Alice's owned e-notes $\{(C_i,K_i^o,m_i)\}_{i=1}^n$, Alice must do the following:
\begin{enumerate}
    \item If the masked amount commitment $C_i'$ is already in ${\tt enimg}_i$ in $T$, then exit this subsection. Else generate $C_i'$ from $(C_i, K_i^o, m_i)$ as per definition, \textit{except} for $i=n$. For the case of $i=n$, set
    $$v_{c,n} = \mathcal{H}_3(q_{ali}) + \mathcal{H}_3(q_{bob}) - \sum_{i=1}^{n-1}{v_{c,i}}.$$
    Note that the value of $v_{c,n}$ is still uniformly random because the values of $t_{c,i}$ for $i\in\{1,\ldots,n-1\}$ are uniformly random.
    \item Insert $C_i'$ to ${\tt enimg}_i$ in $T$.
\end{enumerate}
The generation of $v_{c,n}$ is as such so that the Verifier can verify the amount balance $\sum_{i=1}^n{C_i'}=C_{ali}+C_{bob}$.

\subsection{Membership proofs}\label{mem}
For each of Alice's owned e-notes $\{(C_i,K_i^o,m_i)\}_{i=1}^n$, Alice must do the following:
\begin{enumerate}
    \item If the masked amount commitment $C_i'$ is already in ${\tt enimg}_i$ in $T$, then go to next step. Else generate $C_i'$ from $(C_i, K_i^o, m_i)$ exactly like in Step 1 of Subsection \ref{amt-bal}, and insert it to ${\tt enimg}_i$ in $T$.
    \item If the masked address $K_i'^o$ is already in ${\tt enimg}_i$ in $T$, then go to next step. Else generate $K_i'^o$ from $(C_i, K_i^o, m_i)$ as per definition, and insert it to ${\tt enimg}_i$ in $T$.
    \item Collect $s-1$ number of random e-notes from the ledger and add her owned $(C_i,K_i^o,m_i)$, for a total of $s$ e-notes. The number $s$ is called the \textbf{anonymity size}.
    \item For each e-note in the collection (of size $s$), extract only the amount commitment and one-time address like this: $(C_j, K_j^o)$. Then arrange the $s$ e-notes in random positions. Alice now has an array (of length $s$) of pairs: $\mathbb{S}_i = \{(C_j, K_j^o)\}_{j=1}^s$, which is called the \textbf{ring}. Its elements $(C_j, K_j^o)$ are called the \textbf{ring members}. 
    \item Prepare the proof transcript $\Pi_{\text{mem}, i}$ for a non-interactive proving system for the following relation:
$$\{(G, C_i', K_i'^o \in\mathbb{G}, \mathbb{S}_i\subset\mathbb{G}^2; \pi_i\in\mathbb{N}, t_{c,i}, t_{k,i}\in\mathbb{F}): 1\le\pi_i\le s \wedge C_i' - C_{\pi_i} = t_{c,i} G \wedge K_i'^o - K_{\pi_i}^o = t_{k,i} G \}$$
    \item Append $(\mathbb{S}_i, \Pi_{\text{mem}, i})$ to $({\tt enimg}_i, \ldots)$ in $T$.
\end{enumerate}
The required security properties for application to Seraphis are described in Section \ref{prov-prop}. Specific proving systems satisfying the requirement include CSAG (CLSAG \cite{clsag} without linking) and One-out-of-Many proving system adapted from Groth and Bootle \textit{et al.} \cite{groth, bootle}.

\subsection{Range proofs}\label{range}
For the new e-notes ${\tt enote}_{ali}$ and ${\tt enote}_{bob}$, Alice must do the following:
\begin{enumerate}
    \item Prepare the respective proof transcript $\Pi_{\text{ran}, ali}$ and $\Pi_{\text{ran}, bob}$ for a non-interactive proving system for the following relation:
$$\{(G, H, C \in\mathbb{G}, a_{max}\in\mathbb{F}; x, a\in\mathbb{F}): C = x G + a H \wedge 0\le a \le a_{max}\}$$
	where $a_{max}$ is the maximum e-note amount.
    \item Store $\Pi_{\text{ran}, ali}$ and $\Pi_{\text{ran}, bob}$ to $T$.
\end{enumerate}

The required security properties for application to Seraphis are described in Section \ref{prov-prop}. Specific proving systems satisfying the requirement include Bulletproofs \cite{bp} and Bulletproofs+ \cite{bp-plus}.

\subsection{Receipt}
Once the construction of $T$ is completed, Alice sends it to the network. Its contents must now be
$$T=({\tt enote}_{ali}, {\tt enote}_{bob}, \Pi_{\text{ran}, ali}, \Pi_{\text{ran}, bob}, \{({\tt enimg}_i, \Pi_{\text{o\&u}, i}, \mathbb{S}_i, \Pi_{\text{mem}, i})\}_{i=1}^n).$$
We denote the full construction of $T$ as the function ${\tt tx}(\cdot)$. This function would be used for describing Seraphis security properties (Section \ref{sec-thm}).

Suppose that the Verifier accepts $T$, hence $T$ is now stored in the ledger. When Bob scans the ledger for new transactions, he must do the following for every $T$ he encounters:
\begin{enumerate}
    \item Get a new e-note $(C, K^o, m)$ in $T$. Note that $m$ contains $(R, \overline{a})$ (see the beginning of this whole section).
    \item Compute the nominal sender-receiver shared secret: $q_{nom} = \mathcal{H}_1(k_{bob}^v R)$.
    \item Compute the nominal spend public key: $K_{nom}^s = K^o - \mathcal{H}_2(q_{nom})X$. If $K_{nom}^s = K_{bob}^s$, then the e-note is \textit{connected} to Bob's receiver address, and proceed to the next step (this is the ``connection'' hinted at the beginning of this whole section).  Otherwise (if not equal), the e-note is not connected, and hence go to Step 1.
    \item Decrypt the amount: $a = {\tt dec}[q_{nom}](\overline a)$.
    \item Compute the nominal amount commitment: $C_{nom} = \mathcal{H}_3(q_{nom})G + a H$. If $C_{nom} \ne C$, then the e-note is malformed and cannot be spent.
    \item Compute the nominal linking tag: $\tilde{K}_{nom} = (k_{b, bob}^s/(k_{a, bob}^s + \mathcal{H}_2(q_{nom})))U$. If he finds a copy of $\tilde{K}_{nom}$ in the ledger, then the e-note has already been spent.
\end{enumerate}
If an e-note $(C, K^o, m)$ is connected to Bob's receiver address, then he knows the corresponding scalars of that e-note: $(k_a^o, k_b^o, a, x) = (k_{a, bob}^s + \mathcal{H}_2(q_{nom}), k_{b, bob}^s, a, \mathcal{H}_3(q_{nom}))$. Hence, ``connection'' implies e-note ownership. The transaction is complete for Bob.

For Alice to receive the change e-note, she must do the same above steps. After that, the transcation is complete for Alice. This finishes a Seraphis trancation.

\section{Instantiations and Modifications}\label{inst}
There are a number of details to consider when implementing Seraphis in a real cryptocurrency. Aside from instances of proving systems mentioned already in the previous section, this section is comprised of `recommendations' for instantiations and modifications of other parts of Seraphis, which are inspired by historical privacy-focused cryptocurrency implementations.

\subsection{Addressing schemes}\label{addr-scheme}
\subsection{Multisignature operations}
\subsection{Proof dependency}\label{proof-dep}
\subsubsection*{Transaction Chaining}
\subsection{Transaction fees}
\subsection{Coinbase transactions}
\subsection{Squashed e-note model}
\subsection{?????}

\section{Security model}\label{sec}
For a start, we assume that the distributed ledger is immutable. Therefore, the adversary in our analysis will never be able to modify transactions already stored in the ledger. This ledger immutability can be actualized through, for instance, the Nakamoto consensus protocol \cite{bitcoin}.

Subsections \ref{prov-prop} to \ref{Comm} outline the required security properties of the cryptographic components for Seraphis, then Subsection \ref{sec-thm} is the main security analysis of Seraphis, and Subsection \ref{sec-disc} discusses how some instantiations and modifications described in Section \ref{inst} affect the security analysis.

\subsection{Proving systems security properties}\label{prov-prop}
We define a proving system as a tuple $(\textsf{Setup}, \mathcal{P}, \mathcal{V})$. $\textsf{Setup}$ is the setup algorithm: $pp\leftarrow\textsf{Setup}(1^{\lambda})$, and $\mathcal{P}$ and $\mathcal{V}$ are $\textsf{PPT}$ algorithms called \textbf{Prover} and \textbf{Verifier}, respectively. We denote the \textit{transcript} (all data being sent and received in the protocol) produced by $\mathcal{P}$ and $\mathcal{V}$ when dealing with inputs $x$ and $y$ as $tr\leftarrow\langle\mathcal{P}(x), \mathcal{V}(y)\rangle$. Once the transcript is produced, we denote the final transcript verification done by an algorithm $\mathcal{X}$ as $\mathcal{X}(tr)=1$ for accepted transcript and $\mathcal{X}(tr)=0$ for rejected. 

Let $\mathcal{R}$ be an NP (polynomial-time verifiable) relation of the form $\{(x,w):P(x, w)\}$ where $x$ is the \textit{statement}, $w$ is the \textit{witness}, and $P$ is a predicate of $x$ and $w$. Then ``$(\textsf{Setup}, \mathcal{P}, \mathcal{V})$ is a proving system for the relation $\mathcal{R}$'' informally means that when $\mathcal{P}$ gives an $x$ to $\mathcal{V}$, $\mathcal{P}$ must convince $\mathcal{V}$ that it knows a $w$ such that $(x,w)\in\mathcal{R}$ by generating $tr\leftarrow\langle\mathcal{P}(pp, x, w), \mathcal{V}(pp, x)\rangle$ that $\mathcal{V}$ accepts.

Here are the \textit{minimal} needed security properties of proving systems for Seraphis:

\begin{definition}[Perfect Completeness]
$(\emph{\textsf{Setup}}, \mathcal{P}, \mathcal{V})$ is perfectly complete for $\mathcal{R}$ if for all $\emph{\textsf{PPT}}$ adversary $\mathcal{A}$,
\begin{align*}
\emph{\textsf{Pr}}\left[
\begin{array}{c|c}
    \begin{gathered}
        (x,w)\in\mathcal{R}\\
        \wedge\ \mathcal{V}(tr) = 0
    \end{gathered}
    &
    \begin{gathered}
        pp\leftarrow\emph{\textsf{Setup}}(1^{\lambda}); (x, w)\leftarrow\mathcal{A}(pp); \\
        tr\leftarrow\langle\mathcal{P}(pp,x,w), \mathcal{V}(pp,x)\rangle
    \end{gathered}
\end{array}
\right]
= 0.
\end{align*}
\end{definition}

\begin{definition}[Computational Soundness]
$(\emph{\textsf{Setup}}, \mathcal{P}, \mathcal{V})$ is computationally sound for $\mathcal{R}$ if for all $\emph{\textsf{PPT}}$ adversary $\mathcal{A}$, there exists a negligible function $\emph{\textsf{negl}}(\lambda)$ such that
\begin{align*}
\emph{\textsf{Pr}}\left[
\begin{array}{c|c}
    \begin{gathered}
        (x,w)\not\in\mathcal{R}\\
        \wedge\ \mathcal{V}(tr) = 1
    \end{gathered}
    &
    \begin{gathered}
        pp\leftarrow\emph{\textsf{Setup}}(1^{\lambda}); (x, w)\leftarrow\mathcal{A}(pp); \\
        tr\leftarrow\langle\mathcal{A}(pp,x,w), \mathcal{V}(pp,x)\rangle
    \end{gathered}
\end{array}
\right]
\le \emph{\textsf{negl}}(\lambda).
\end{align*}
\end{definition}

There is another notion of soundness called \textit{Special Soundness}. For a proving system to be special sound, there must exist a \textit{witness extractor} that has an ability to ``rewind time'' and make the Prover answer several different challenges, and it must be able to extract a witness given the several accepted transcripts with the Prover. Because of this, proving systems satisfying special soundness are called \textit{Proofs of Knowledge}. Special soundness is a stronger notion of soundness, hence this already implies computational soundness.

\begin{definition}[Perfect Special HVZK]
$(\emph{\textsf{Setup}}, \mathcal{P}, \mathcal{V})$ is perfectly special honest-verifier zero knowledge (HVZK) for $\mathcal{R}$ if there exists a $\emph{\textsf{PPT}}$ simulator $\mathcal{S}$ such that for all $\emph{\textsf{PPT}}$ adversary $\mathcal{A}$,
\begin{align*}
\emph{\textsf{Pr}}\left[
\begin{array}{c|c}
    \begin{gathered}
        (x,w)\in\mathcal{R}\\
        \wedge\ \mathcal{A}(tr) = 1
    \end{gathered}
    &
    \begin{gathered}
        pp\leftarrow\emph{\textsf{Setup}}(1^{\lambda}); (x, w, \rho)\leftarrow\mathcal{A}(pp); \\
        tr\leftarrow\langle\mathcal{P}(pp,x,w), \mathcal{V}(pp,x; \rho)\rangle
    \end{gathered}
\end{array}
\right] \\
=\emph{\textsf{Pr}}\left[
\begin{array}{c|c}
    \begin{gathered}
        (x,w)\in\mathcal{R}\\
        \wedge\ \mathcal{A}(tr) = 1
    \end{gathered}
    &
    \begin{gathered}
        pp\leftarrow\emph{\textsf{Setup}}(1^{\lambda}); (x, w, \rho)\leftarrow\mathcal{A}(pp); \\
        tr\leftarrow\mathcal{S}(pp, x, \rho)
    \end{gathered}
\end{array}
\right]
\end{align*}
where $\rho$ is the public randomness used by $\mathcal{V}$.
\end{definition}

Fiat-Shamir heuristic \cite{fiat-shamir} is applied to make interactive proving systems non-interactive. Moreover, Fiat-Shamir heuristic transforms interactive protocols satisfying HVZK into non-interactive (fully) zero-knowledge (NIZK) protocols in the random oracle model. Hence, all proving systems for application to Seraphis should be transformed via Fiat-Shamir heuristic.

For membership proofs, we need the following property which is weaker than perfect special HVZK \cite{groth}:
\begin{definition}[Witness Indistinguishability]
$(\emph{\textsf{Setup}}, \mathcal{P}, \mathcal{V})$ is witness indistinguishable for $\mathcal{R}$ if for all $\emph{\textsf{PPT}}$ adversary $\mathcal{A}$, there exists a negligible function $\emph{\textsf{negl}}(\lambda)$ such that
\begin{align*}
\left| \frac{1}{2} - \emph{\textsf{Pr}}\left[
\begin{array}{c|c}
    \begin{gathered}
        b' = b
    \end{gathered}
    &
    \begin{gathered}
        pp\leftarrow\emph{\textsf{Setup}}(1^{\lambda}); (x, w_0, w_1)\leftarrow\mathcal{A}(pp); \\
        b\xleftarrow{\$}\{0,1\}; tr\leftarrow\langle\mathcal{P}(pp,x,w_b), \mathcal{V}(pp,x)\rangle; \\
        b'\leftarrow\mathcal{A}(tr)
    \end{gathered}
\end{array}
\right]\right|
\le\emph{\textsf{negl}}(\lambda)
\end{align*}
where $\mathcal{A}(pp)$ always outputs $(x,w_0,w_1)$ such that $(x,w_0)\in\mathcal{R}\wedge(x,w_1)\in\mathcal{R}$.
\end{definition}

All proving systems for application to Seraphis must \textit{at least} have perfect completeness and computational soundness. Moreover, ownership and unspentness proofs and range proofs must at least have perfect special HVZK, while membership proofs must at least have witness indistinguishability.

\subsection{Authenticated symmetric encryption scheme}\label{sec-symm}
We require that the authenticated symmetric encryption scheme must at least have the following properties: indistinguishable against adaptive chosen-ciphertext attack (IND-CCA2) and key-private under chosen-ciphertext attacks (IK-CCA). These properties are defined in the Appendix A.4 of \cite{omniring}.

\subsection{Commitment schemes}\label{Comm}

We define a commitment scheme as a tuple $(\textsf{Setup}, \textsf{Comm})$. $\textsf{Setup}$ is the setup algorithm: $pp\leftarrow\textsf{Setup}(1^{\lambda})$, and $\textsf{Comm}:\mathcal{M}\times{\chi}\rightarrow\mathcal{C}$ is the $\textit{commitment function}$, where $\mathcal{M}$ is the message space, $\chi$ is the randomness space, and $\mathcal{C}$ is the commitment space. Note that $\mathcal{M}, \chi$ and $\mathcal{C}$ are all constructed from $pp$.
To commit to a message $m \in M$, the sender selects $r\xleftarrow{\$}\chi$ and computes the commitment $C = \textsf{Comm}(m; r)$. We define the required security properties of commitment schemes.

\begin{definition}[Hiding Property]
A commitment scheme $(\emph{\textsf{Setup}}, \emph{\textsf{Comm}})$ is computationally hiding if for all $\emph{\textsf{PPT}}$ adversary $\mathcal{A}$, there exists a negligible function $\emph{\textsf{negl}}(\lambda)$ such that
\begin{align*}
\left| \frac{1}{2} - \emph{\textsf{Pr}}\left[
\begin{array}{c|c}
    \begin{gathered}
        b' = b
    \end{gathered}
    &
    \begin{gathered}
        pp\leftarrow\emph{\textsf{Setup}}(1^{\lambda}); (m_0, m_1)\leftarrow\mathcal{A}(pp); \\
        b\xleftarrow{\$}\{0,1\}; r \xleftarrow{\$}\chi; \\
        C = \emph{\textsf{Comm}}(m_b; r); b'\leftarrow\mathcal{A}(C)
    \end{gathered}
\end{array}
\right]\right|
\le \emph{\textsf{negl}}(\lambda).
\end{align*}
\end{definition}
A commitment scheme is \textit{perfectly hiding} if $\textsf{negl}(\lambda)$ is replaced by $0$.

\begin{definition}[Binding Property]
A commitment scheme $(\emph{\textsf{Setup}}, \emph{\textsf{Comm}})$ is computationally binding if for all $\emph{\textsf{PPT}}$ adversary $\mathcal{A}$, there exists a negligible function $\emph{\textsf{negl}}(\lambda)$ such that
\begin{align*}
\emph{\textsf{Pr}}
\left[
\begin{array}{c|c}
    \begin{gathered}
         \emph{\textsf{Comm}}(m_0;r_0) \\
        = \emph{\textsf{Comm}}(m_1;r_1) \\
        \wedge\ m_0 \ne m_1
    \end{gathered}
    &
    \begin{gathered}
        pp\leftarrow\emph{\textsf{Setup}}(1^{\lambda}); \\
        (m_0,m_1,r_0,r_1)\leftarrow\mathcal{A}(pp)
    \end{gathered}
\end{array}
\right]
\le \emph{\textsf{negl}}(\lambda).
\end{align*}
\end{definition}
A commitment scheme is \textit{perfectly binding} if $\textsf{negl}(\lambda)$ is replaced by $0$.

\noindent The first kind of commitment we define is commonly known as \textbf{Pedersen commitments} \cite{pedersen}. We define two instances, $\textsf{PedersenC}:\mathbb{F}\times\mathbb{F}\rightarrow\mathbb{G}$ and $\textsf{PedersenK}:\mathbb{F}\times\mathbb{F}\rightarrow\mathbb{G}$ as follows:
\begin{align*}
\textsf{PedersenC}(a; x) &= x G + a H \ , \ x\xleftarrow{\$}\mathbb{F}\ \text{if}\ x\ \text{is not specified.} \\
\textsf{PedersenK}(k_b^o;k_a^o) &= k_a^o X + k_b^o U \ , \ k_a^o\xleftarrow{\$}\mathbb{F}\ \text{if}\ k_a^o\ \text{is not specified.}
\end{align*}
$\textsf{PedersenC}$ corresponds to the structure of amount commitment $C$ and masked amount commitment $C'$, while $\textsf{PedersenK}$ corresponds to the structure of one-time address $K^o$.
\begin{theorem}[From \cite{pedersen}]\label{thm-pedersen}
Pedersen commitment is perfectly hiding and computationally binding under the DL assumption.
\end{theorem}

\noindent Then we define a custom commitment $\textsf{LinkTag}:\mathbb{F}\setminus\{0\}\times\mathbb{F}\times\mathbb{F}\rightarrow\mathbb{G}\times\mathbb{G}$ as follows:
\begin{align*}
\textsf{LinkTag}(k_a, k_b; t_k) = (t_k G + k_a X + k_b U, (k_b/k_a)U)\ , \ t_k\xleftarrow{\$}\mathbb{F}\ \text{if}\ t_k\ \text{is not specified.}
\end{align*}
$\textsf{LinkTag}$ corresponds to structure of the combination of masked address $K'^o$ and linking tag $\tilde{K}$.

\begin{theorem}\label{thm-linktag}
$\emph{\textsf{LinkTag}}$ is perfectly hiding and computationally binding under the DL assumption.
\end{theorem}
\noindent We prove this in Appendix \ref{proofs}.

\subsection{Seraphis security properties}\label{sec-thm}
The required security properties for Seraphis are based on Omniring's security model \cite{omniring}. The Omniring paper presents a rigorous formalization of RingCT constructions, providing precision for security analysis against several realistic attacks.

It is important to note that the properties in subsection \ref{prov-prop} are the \textit{minimal} requirements, thus the properties presented in this subsection are the \textit{weakest}. Because of Seraphis's modularity, stronger properties for proving systems must yield stronger properties for Seraphis. Proofs for the theorems in this subsection, found in Appendix \ref{proofs}, can also serve a guide for proving stronger properties.

Lastly, note that the mentioned proving systems in the theorems are in their ``original'' interactive versions. As Fiat-Shamir heuristic produces full zero-knowledge which is stronger than HVZK, the theorems should also apply to the non-interactive protocols transformed through the heuristic.

The first security property is \textbf{Completeness} (called \textit{Correctness} in Omniring), which means that if an e-note appears on the ledger, then the user owning it can honestly generate an accepted transaction spending it. Seraphis satisfying completeness immediately follows from the completeness properties of the cryptographic components and by inspection of the protocol description.

Next we consider the \textbf{Balance} property, which means that a spender adversary should never be able to spend more amounts than he truly owns, hence preventing double-spending. The one presented here would be weaker than the one in Omniring because the one in Omniring requires special soundness for proving systems, which we do not require. Balance property involves an experiment $\textsf{BAL}(\mathcal{A}, 1^{\lambda})$  on a $\textsf{PPT}$ adversary $\mathcal{A}$. The adversary succeeds in the experiment (i.e. $\textsf{BAL}(\mathcal{A}, 1^{\lambda})=1$) if he managed to generate an accepted transaction such that 1) some of spent e-note one-time private keys are \textit{not equal} to what is originally given to him, leading to a double-spend of same e-notes under different linking tags, or 2) the amount of the new e-note for the receiver is \textit{larger} than the supposed total amount of e-notes he owns.

\begin{figure}[htbp]
\centering
\fbox{\begin{minipage}{0.85\textwidth}
\underline{$\textsf{BAL}(\mathcal{A}, 1^{\lambda})$}
\begin{itemize}
\item $pp\leftarrow\textsf{Setup}(1^{\lambda})$.
\item $\mathcal{A}$ is provided random $k^v, k_a^s, k_b^s\in\mathbb{F}$ to construct the address ${\tt addr}_{\mathcal{A}} = (K_{\mathcal{A}}^{dh}, K_{\mathcal{A}}^v, K_{\mathcal{A}}^s)$, and $\{(k_{a,i}^o, k_{b,i}^o, a_i, x_i)\}_{i=1}^n$ that makes ${\tt addr}_{\mathcal{A}}$ connect to ${\tt enote}_{\mathcal{A}}=\{(C_i,K_i^o,m_i)\}_{i=1}^n$ in the ledger.
\item $\mathcal{A}$ chooses any receiver address ${\tt addr}_{\mathcal{B}} = (K_{\mathcal{B}}^{dh}, K_{\mathcal{B}}^v, K_{\mathcal{B}}^s)$.
\item $\{(k_{a,i}'^o, k_{b,i}'^o, a'_i, x'_i)\}_{i=1}^n \leftarrow\mathcal{A}(pp, {\tt enote}_{\mathcal{A}}, \{(k_{a,i}^o, k_{b,i}^o, a_i, x_i)\}_{i=1}^n)$.
\item $T\leftarrow{\tt tx}(pp, {\tt addr}_{\mathcal{A}}, {\tt addr}_{\mathcal{B}}, {\tt enote}_{\mathcal{A}}, \{(k_{a,i}'^o, k_{b,i}'^o, a'_i, x'_i)\}_{i=1}^n)$. $\mathcal{A}$ spends all e-notes in ${\tt enote}_{\mathcal{A}}$ to send an amount $a_t$ to ${\tt addr}_{\mathcal{B}}$.
\item $b_0:=1$ if Verifier accepts $T$, else $:=0$.
\item $b_1:=1$ if $\exists i\in\{1,\ldots,n\}: (k_{a,i}'^o, k_{b,i}'^o) \ne (k_{a,i}^o, k_{b,i}^o)$, else $:=0$.
\item $b_2:=1$ if $\sum_{i=1}^n{a_i} < a_t$, else $:=0$.
\item Return $b_0 \wedge (b_1 \vee b_2)$.
\end{itemize}
\end{minipage}}
\caption{Balance experiment $\textsf{BAL}$}
\label{exp-bal}
\end{figure}

\begin{definition}[Balance]
Seraphis is balanced if for all $\emph{\textsf{PPT}}$ adversary $\mathcal{A}$, there exists a negligible function $\emph{\textsf{negl}}(\lambda)$ such that
\begin{align*}
\emph{\textsf{Pr}}
\left[
\emph{\textsf{BAL}}(\mathcal{A}, 1^{\lambda})=1
\right]
\le \emph{\textsf{negl}}(\lambda).
\end{align*}
where $\emph{\textsf{BAL}}$ is described in Figure \ref{exp-bal}.
\end{definition}
\begin{theorem}[Balance]\label{thm-bal}
If $(\emph{\textsf{Setup}},\emph{\textsf{PedersenC}})$ and $(\emph{\textsf{Setup}},\emph{\textsf{LinkTag}})$ are both binding, and all proving systems are (computational) sound, then Seraphis is balanced.  
\end{theorem}
Next we consider the \textbf{Privacy} property, which means that an adversary should never be able to detect the spender, receiver, and amounts in any transaction, hence providing sender and receiver anonymity, and confidential transfer of amounts. Privacy property involves an experiment $\textsf{PRV}(\mathcal{A}, 1^{\lambda})$ on a $\textsf{PPT}$ adversary $\mathcal{A}$. In the experiment, it is as if $\mathcal{A}$ himself \textit{sent} amounts to the two potential senders, hence he is provided the sender addresses, the e-notes themselves, and the private scalars of the amount commitment $C$ in those e-notes. Given a whole transaction $T$, the adversary succeeds in the experiment if she can guess the sender, the receiver, or the amount of the new e-note for the receiver, hence breaking the privacy of $T$.

\begin{figure}[htbp]
\centering
\fbox{\begin{minipage}{0.85\textwidth}
\underline{$\textsf{PRV}(\mathcal{A}, 1^{\lambda})$}
\begin{itemize}
\item $pp\leftarrow\textsf{Setup}(1^{\lambda})$.
\item $\mathcal{A}$ is provided two random potential sender addresses ${\tt send}_0$ and ${\tt send}_1$, sets of e-notes ${\tt enote}_0$ and ${\tt enote}_1$ (with $|{\tt enote}_0|=|{\tt enote}_1|=n$) connected to ${\tt send}_0$ and ${\tt send}_1$ respectively, private scalars of $C$, $\{(x_{0, i}, a_{0, i})\}_{i=1}^n$ and $\{(x_{1, i}, a_{1, i})\}_{i=1}^n$, of each e-note in ${\tt enote}_0$ and ${\tt enote}_1$, respectively, and two random potential receiver addresses ${\tt recv}_0$ and ${\tt recv}_1$.
\item $\mathcal{A}$ constructs $\{\mathbb{S}_i\}_{i=1}^n$ such that each $\mathbb{S}_i$ contains one e-note in ${\tt enote}_0$ and one e-note in ${\tt enote}_1$.
\item $\mathcal{A}$ chooses any valid amount the potential senders would send: $0 \le a_{\mathcal{A}, 0} \le \sum_{i=1}^n{a_{0, i}}$ and $0 \le a_{\mathcal{A}, 1} \le \sum_{i=1}^n{a_{1, i}}$ for ${\tt send}_0$ and ${\tt send}_1$, respectively.
\item $b\xleftarrow{\$}\{0,1\}$.
\item $T\leftarrow{\tt tx}(pp, {\tt send}_b, {\tt recv}_b, \{\mathbb{S}_i\}_{i=1}^n, a_{\mathcal{A}, b})$. The user who owns ${\tt send}_b$ spends the e-notes they own in $\{\mathbb{S}_i\}_{i=1}^n$ to send the amount $a_{\mathcal{A}, b}$ to ${\tt recv}_b$.
\item If Verifier rejects $T$, then return $0$.
\item $b'\leftarrow\mathcal{A}(pp, T, \{( {\tt send}_j, {\tt recv}_j, \{(x_{j, i}, a_{j,i})\}_{i=1}^n, a_{\mathcal{A},j})\}_{j\in \{0,1\}})$.
\item Return $1$ if $b = b'$, else $0$.
\end{itemize}
\end{minipage}}
\caption{Privacy experiment $\textsf{PRV}$}
\label{exp-prv}
\end{figure}

\begin{definition}[Privacy]
Seraphis is private if for all $\emph{\textsf{PPT}}$ adversary $\mathcal{A}$, there exist a negligible function $\emph{\textsf{negl}}(\lambda)$ such that
\begin{align*}
\emph{\textsf{Pr}}
\left[
\emph{\textsf{PRV}}(\mathcal{A}, 1^{\lambda})=1
\right]
\le \emph{\textsf{negl}}(\lambda).
\end{align*}
where $\emph{\textsf{PRV}}$ is described in Figure \ref{exp-prv}.
\end{definition}
\begin{theorem}[Privacy]\label{thm-prv}
If $(\emph{\textsf{Setup}},\emph{\textsf{PedersenC}})$, $(\emph{\textsf{Setup}},\emph{\textsf{PedersenK}})$ and $(\emph{\textsf{Setup}},\emph{\textsf{LinkTag}})$ are all hiding, ownership and unspentness proof and range proof are both perfect special HVZK, membership proof is witness indistinguishable, and authenticated symmetric encryption scheme is IND-CCA2 and IK-CCA, then Seraphis is private.  
\end{theorem}

Lastly, we consider the \textbf{Non-slanderability} property, which means that an adversary should never be able to forge a linking tag of an honest user's e-notes when those would be spent. This property prevents the following attack known as \textit{denial-of-spending attack} \cite{denial-of-spend}: when the attacker first spends their e-notes with forged linking tags, a victim user owning e-notes with matching linking tags cannot spend their e-notes anymore because a match in linking tag implies a double-spend. Non-slanderability property involves an experiment $\textsf{NSLAND}(\mathcal{A}, 1^{\lambda})$ on a $\textsf{PPT}$ adversary $\mathcal{A}$. The adversary succeeds in the experiment if they successfully accomplished a denial-of-spending attack.

\begin{figure}[htbp]
\centering
\fbox{\begin{minipage}{0.85\textwidth}
\underline{$\textsf{NSLAND}(\mathcal{A}, 1^{\lambda})$}
\begin{itemize}
\item $pp\leftarrow\textsf{Setup}(1^{\lambda})$.
\item $\mathcal{A}$ is provided random $k^v, k_a^s, k_b^s\in\mathbb{F}$ to construct the address ${\tt addr}_{\mathcal{A}} = (K_{\mathcal{A}}^{dh}, K_{\mathcal{A}}^v, K_{\mathcal{A}}^s)$, and ${\tt enote}_{\mathcal{A}}=\{(C_i,K_i^o,m_i)\}_{i=1}^n$ in the ledger connected to ${\tt addr}_{\mathcal{A}}$. $\mathcal{A}$ is also provided a random victim address ${\tt addr}_{\mathcal{C}} = (K_{\mathcal{C}}^{dh}, K_{\mathcal{C}}^v, K_{\mathcal{C}}^s)$, and ${\tt enote}_{\mathcal{C}}$, which is all the e-notes in the ledger connected to ${\tt addr}_{\mathcal{C}}$. Let $\{\tilde{K}_{\mathcal{C}}\}$ be all the would-be linking tags of e-notes in ${\tt enote}_{\mathcal{C}}$ when those are spent honestly by the owner of ${\tt addr}_{\mathcal{C}}$.
\item $\mathcal{A}$ chooses any receiver address ${\tt addr}_{\mathcal{B}} = (K_{\mathcal{B}}^{dh}, K_{\mathcal{B}}^v, K_{\mathcal{B}}^s)$.
\item $\{(k_{a,i}'^o, k_{b,i}'^o, a'_i, x'_i)\}_{i=1}^n \leftarrow\mathcal{A}(pp, (k'^v, k_a'^s, k_b'^s), {\tt addr}_{\mathcal{B}},  {\tt addr}_{\mathcal{C}}, {\tt enote}_{\mathcal{A}}, {\tt enote}_{\mathcal{C}})$.
\item $T\leftarrow{\tt tx}(pp, {\tt addr}_{\mathcal{A}}, {\tt addr}_{\mathcal{B}}, {\tt enote}_{\mathcal{A}}, \{(k_{a,i}'^o, k_{b,i}'^o, a'_i, x'_i)\}_{i=1}^n)$. $\mathcal{A}$ spends all e-notes in ${\tt enote}_{\mathcal{A}}$ to send some amounts to ${\tt addr}_{\mathcal{B}}$. Let $\{\tilde{K}_{\mathcal{B}}\}$ be the linking tags of all spent e-notes in $T$.
\item $b_0:=1$ if Verifier accepts $T$, else $:=0$
\item $b_1:=1$ if $\{\tilde{K}_{\mathcal{B}}\}\cap \{\tilde{K}_{\mathcal{C}}\}\ne\emptyset$, else $:=0$
\item Return $b_0 \wedge b_1$.
\end{itemize}
\end{minipage}}
\caption{Non-slanderability experiment $\textsf{NSLAND}$}
\label{exp-nsland}
\end{figure}

\begin{definition}[Non-slanderability]
Seraphis is non-slanderable if for all $\emph{\textsf{PPT}}$ adversary $\mathcal{A}$, there exist a negligible function $\emph{\textsf{negl}}(\lambda)$ such that
\begin{align*}
\emph{\textsf{Pr}}
\left[
\emph{\textsf{NSLAND}}(\mathcal{A}, 1^{\lambda})=1
\right]
\le \emph{\textsf{negl}}(\lambda).
\end{align*}
where $\emph{\textsf{NSLAND}}$ is described in Figure \ref{exp-nsland}.
\end{definition}
\begin{theorem}[Non-slanderability]\label{thm-nsland}
If $(\emph{\textsf{Setup}},\emph{\textsf{LinkTag}})$ is hiding, all proving systems are (computational) sound, ownership and unspentness proof and range proof are both perfect special HVZK, and membership proof is witness indistinguishable, then Seraphis is non-slanderable.  
\end{theorem}
Non-slanderability and the binding of linking tags to a unique spent e-note one-time private key implies that an adversary should never be able to forge an accepted transaction of another honest user, a property known as \textbf{Unforgeability}. Since the binding property of $(\textsf{Setup},\textsf{LinkTag})$ is already required in the balance property, there is no need to formally define and prove an unforgeability property.

\subsection{Discussions}\label{sec-disc}

\section{Efficiency}

\bibliographystyle{plain}
\bibliography{seraphis}

\appendix

\section{Composition proving system}\label{comp-prov}
The composition proving system is a protocol for the relation:
\begin{multline*}
\Big\{\big(G, X, U\in\mathbb{G}, \{K_i\}_{i=1}^n, \{K_{t1,i}\}_{i=1}^n, \{\tilde{K}_i\}_{i=1}^n \in\mathbb{G}^n; \{x_i\}_{i=1}^n, \{y_i\}_{i=1}^n, \{z_i\}_{i=1}^n \in\mathbb{F}^n\big): \\ \bigwedge_{i=1}^n{\big(y_i \ne 0 \wedge K_i = x_i G + y_i X + z_i U \wedge K_{t1,i} = (1/y_i)K_i \wedge \tilde{K}_i = (z_i/y_i)U\big)} \Big\}
\end{multline*}
Now the Prover only needs to produce one proof transcript for all $i$'s instead of one proof transcript for each $i$. This protocol is based on the concise approach from \cite{clsag} to reduce proof sizes when constructing multiple proofs in parallel. Notice the extra $\{K_{t1,i}\}_{i=1}^n$ in the relation. This should not affect the proving system's applicability for ownership and unspentness proof because all $y_i$'s are still hidden in $K_{t1,i}$ (by the DL assumption) and the required relationships for $\{K_i\}_{i=1}^n$ and $\{\tilde{K}_i\}_{i=1}^n$ are still proven.

The protocol proceeds as follows:
\begin{enumerate}
\item The Prover generates $\alpha_a, \alpha_b, \alpha_i \xleftarrow{\$}\mathbb{F}\ ,\ \forall i\in\{1,\ldots,n\}$. The Prover computes $(A_a, A_b) = (\alpha_a G, \alpha_b U)$ and $A_{i}=\alpha_i K_i\ ,\ \forall i\in\{1,\ldots,n\}$, and sends $(A_a, A_b, \{A_i\}_{i=1}^n)$ to the verifier.
\item Both the Prover and Verifier compute $K_{t2,i} = K_{t1,i} - X - \tilde{K}_i\ ,\ \forall i\in\{1,\ldots,n\}$.
\item The Verifier sends a challenge $c\xleftarrow{\$}\mathbb{F}$ and random scalars $\mu_a,\mu_b\xleftarrow{\$}\mathbb{F}$ to the Prover.
\item The Prover computes the responses:
\begin{align*}
r_{a} &= \alpha_a - c \sum_{i=1}^{n}{\mu_a^{i-1} (x_i/y_i)} \\
r_{b} &= \alpha_b - c \sum_{i=1}^{n}{\mu_b^{i-1}(z_i/y_i)} \\
r_{i} &= \alpha_i - c (1/y_i)\ ,\ \forall i\in\{1,\ldots,n\}\\
\end{align*}
and sends those values to the Verifier.
\item The Verifier checks the following equalities. If any of them fail, then the Prover has failed to satisfy the composition proof system.
\begin{align*}
A_{a} &= r_a G + c \sum_{i=1}^{n}{\mu_a^{i-1} K_{t2,i}} \\
A_{b} &= r_b U + c \sum_{i=1}^{n}{\mu_b^{i-1} \tilde{K}_i} \\
A_{i} &= r_i K_i + c K_{t1,i}\ ,\ \forall i\in\{1,\ldots,n\}\\
\end{align*}
\end{enumerate}
We now prove that the protocol is perfectly complete, computationally sound, and perfectly special honest-verifier zero knowledge, all if $G$, $X$, and $U$ are mutually independent.

\begin{proof}
For perfect completeness, note that $\forall i\in\{1,\ldots,n\}$, knowledge of $(x_i/y_i$, $z_i/y_i$, $1/y_i)$ is also enough to satisfy the proving relation, and
\begin{align*}
K_{t2,i} &= K_{t1,i} - X - \tilde{K}_i \\ &= (1/y_i)(x_i G + y_i X + z_i U) - X - (z_i/y_i)U \\
&= (x_i/y_i)G + X + (z_i/y_i) U - X - (z_i/y_i)U \\ &= (x_i/y_i)G.
\end{align*}
Then the property follows from inspection.

For perfect special HVZK, we construct a simulator producing accepted transcripts with probability distribution identical to the probability distribution of legitimate accepted transcripts by Prover and Verifier. The simulator generates $c, \mu_a, \mu_b, r_a, r_b, r_i \xleftarrow{\$}\mathbb{F}\ ,\ \forall i\in\{1,\ldots,n\}$. Then the simulator computes $A_{a}, A_{b}, \{A_{i}\}_{i=1}^n$ exactly according to Step 5 of the protocol description. Because of this last step, the Verifier will accept the simulated transcript. Now assume that $G$, $X$, and $U$ are mutually independent. Observe that the simulated transcript is uniformly random because $c, \mu_a, \mu_b, r_a, r_b, \{r_i\}_{i=1}^n$ are uniformly randomly selected. On the other hand, observe that a legitimate accepted transcript between Prover and Verifier is also uniformly random because the Prover's $\alpha_a, \alpha_b, \{\alpha_i\}_{i=1}^n$ and the Verifier's $c, \mu_a, \mu_b$ are all uniformly randomly selected. Hence the two probability distributions are identical.

For computational soundness under the DL assumption, the proof is by contraposition. Suppose that a $\textsf{PPT}$ adversarial prover $\mathcal{A}$ \textit{not} knowing a witness, can produce an accepted transcript $$(A_a, A_b, \{A_i\}_{i=1}^n,c, \mu_a, \mu_b, r_a, r_b, \{r_i\}_{i=1}^n)$$ with an honest verifier with non-negligible probability. From the third equation in Step 5, $\forall i\in\{1,\ldots,n\}$:
\begin{align*}
A_{i} &= r_i K_i + c K_{t1,i} \\
\implies\alpha_i K_i &= r_i K_i + c (1/y_i)K_i \\
\implies\alpha_i &= r_i + c(1/y_i) \\
\implies(1/y_i) &= (\alpha_i - r_i)(1/c)
\end{align*}
Hence solving the discrete log (DL) problem for $K_{t1,i} = (1/y_i)K_i$ with non-negligible probability. Note that $\alpha_i$ and $(1/y_i)$ must always exist because $\mathbb{G}$ is of prime order which implies that any non-zero point is a generator.
\end{proof}

\begin{remark}
{\normalfont
Since only computational soundness is proven for this protocol, this cannot be called a proof of knowledge. As far as the author's knowledge, constructing a composition proving system that satisfies special soundness is an open problem.}
\end{remark}

Fiat-Shamir heuristic transforms interactive protocols satisfying HVZK into non-interactive (fully) zero-knowledge (NIZK) protocols in the random oracle model. Applying Fiat-Shamir heuristic to the composition proving system should be straightforward. Note that the $c$, $\mu_a$ and $\mu_b$ are generated through domain separation of hash function $\mathcal{H}$.

\section{Proofs of some theorems in Section \ref{sec}}\label{proofs}
We first present a lemma which may be helpful in proofs by reduction.

\begin{definition}\label{negl}
A function $f:\mathbb{N}\rightarrow\mathbb{R}$ is \textbf{\em negligible} if for all polynomial $p(\cdot)$ there exists an $N\in\mathbb{N}$ such that for all integers $n>N$ it holds that $f(n)<\frac{1}{p(n)}$.
\end{definition}
\noindent Definition \ref{negl} is from Katz \& Lindell \cite{katz-lindell}. We prove the following lemma:

\begin{lemma}\label{negl-exp}
If $f:\mathbb{N}\rightarrow\mathbb{R}$ is non-negligible, then $g(\cdot)=f(\cdot)^m$ for any $m\in\mathbb{N}$ and $m>1$ is non-negligible.
\end{lemma}
\begin{proof}
The non-negligibility of $f$ means that there exists a polynomial $p(\cdot)$ such that for all $N\in\mathbb{N}$, there exists an $n>N$ such that $f(n)\ge\frac{1}{p(n)}$. Let $p_f(\cdot)$ be such polynomial and $n_f$ be such $n>N$. Then setting $p_g(\cdot)=p_f(\cdot)^m$ and $n_g=n_f$ suffices for non-negligibility of $g$ because $f(n_f)\ge\frac{1}{p_f(n_f)}\implies f(n_f)^m\ge\frac{1}{p_f(n_f)^m}$.
\end{proof}
\noindent Lemma \ref{negl-exp} justifies the usage of finite number of ``breaks'' of one hardness assumption in proofs by reduction. For a start, the probability of breaking the hardness assumption $\textsf{HA}$ is a function of $\lambda$. Just here we denote this as $\textsf{Pr}[\textsf{HA}(\lambda)]$. Hence, for $m>1$, the probability for breaking $\textsf{HA}$ $m$-times, $\textsf{Pr}[\wedge_{i=1}^{m}{\textsf{HA}_i(\lambda)}]\ge\textsf{Pr}[\textsf{HA}(\lambda)]^m$. Now Lemma \ref{negl-exp} says that if $\textsf{Pr}[\textsf{HA}(\lambda)]$ is non-negligible (or equivalently, for all negligible function $\textsf{negl}(\lambda)$, $\textsf{Pr}[\textsf{HA}(\lambda)]\ge\textsf{negl}(\lambda)$), then $\textsf{Pr}[\textsf{HA}(\lambda)]^m$ must also be non-negligible and hence $\textsf{Pr}[\wedge_{i=1}^{m}{\textsf{HA}_i(\lambda)}]$ is also non-negligible. Nevertheless, the number of breaks should still be documented in proofs.

\begin{proof}[Proof of Theorem \ref{thm-linktag}]
Perfect hiding follows from the proof of perfect hiding of Pedersen commitments because the blinding factor $t_k$ is only used in one of the pair output: the one corresponding to $K'^o$.

Computational binding is proven by contraposition: if $\mathcal{A}$ can find $(k_a, k_b, t_k)$ and $(k'_a, k'_b, t'_k)$ such that the two are \textit{not} equal and $\textsf{LinkTag}(k_a, k_b; t_k) = \textsf{LinkTag}(k'_a, k'_b; t'_k)$, then $\mathcal{A}$ can find a DL relationship among $G,X,U\in\mathbb{G}$ which (this is very important) supposedly have unknown DL relationship to each other. Assume that $\mathcal{A}$ found one such $(k_a, k_b, t_k)$ and $(k'_a, k'_b, t'_k)$. Then $t_k G + k_a X + k_b U = t'_k G + k'_a X + k'_b U$ and $(k_b/k_a)U = (k'_b/k'_a)U$ implies that 1) $(t_k - t'_k)G + (k_a - k'_a)X + (k_b - k'_b)U = 0$ and 2) $k'_a k_b = k_a k'_b$. We will focus more on equation 1 because equation 2 only adds a restriction to the solution space. Now by Lemma \ref{negl-exp}, $\mathcal{A}$ can find another $(k_{a,2}, k_{b,2}, t_{k,2})$ and $(k'_{a,2}, k'_{b,2}, t'_{k,2})$ (for a total of 2 breaks) such that the two are not equal and $\textsf{LinkTag}(k_{a,2}, k_{b,2}; t_{k,2}) = \textsf{LinkTag}(k'_{a,2}, k'_{b,2}; t'_{k,2})$. The second version of equation 1 would be $(t_{k,2} - t'_{k,2})G + (k_{a,2} - k'_{a,2})X + (k_{b,2} - k'_{b,2})U = 0$. Multiplying $(k_{b,2} - k'_{b,2})$ to the first version of equation 1, multiplying  $(k_b - k'_b)$ to the second version of equation 1, and subtracting the two equations together will yield
\begin{gather*}
\begin{multlined}
[(k_{b,2} - k'_{b,2})(t_k - t'_k)G + (k_{b,2} - k'_{b,2})(k_a - k'_a)X + (k_{b,2} - k'_{b,2})(k_b - k'_b)U] \\ -\ [(k_b - k'_b)(t_{k,2} - t'_{k,2})G + (k_b - k'_b)(k_{a,2} - k'_{a,2})X + (k_b - k'_b)(k_{b,2} - k'_{b,2})U] = 0
\end{multlined} \\
\implies [(k_{b,2} - k'_{b,2})(t_k - t'_k)- (k_b - k'_b)(t_{k,2} - t'_{k,2})]G + [(k_{b,2} - k'_{b,2})(k_a - k'_a) - (k_b - k'_b)(k_{a,2} - k'_{a,2})]X = 0 \\
\implies X = \Bigg[-\frac{(k_{b,2} - k'_{b,2})(t_k - t'_k)- (k_b - k'_b)(t_{k,2} - t'_{k,2})}{(k_{b,2} - k'_{b,2})(k_a - k'_a) - (k_b - k'_b)(k_{a,2} - k'_{a,2})}\Bigg]G
\end{gather*}
which gives the DL relationship between $G$ and $X$.
\end{proof}
\begin{proof}[Proof of Theorem \ref{thm-bal}]
We prove by contraposition. Assume that $\mathcal{A}$ succeeds in the $\textsf{BAL}$ experiment with non-negligible probability. There are two cases as to \textit{why} $\mathcal{A}$ succeeded, each with sub-cases: $b_0 \wedge b_1 = 1$ or $b_0 \wedge b_2 = 1$.

For the first case, assume that $b_0 \wedge b_1 = 1$. One sub-case is that there exists $i\in\{1,\ldots,n\}$ such that in the e-note image, $K_i'^o = t_{k,i} G + k_{a,i}^o X + k_{b,i}^o U = t'_{k,i} G + k_{a,i}'^o X + k_{b,i}'^o U$ (where $t'_{k,i}\in\mathbb{F}$ is also computed by $\mathcal{A}$) and $\tilde{K}_i = (k_{b,i}^o/k_{a,i}^o)U = (k_{b,i}'^o/k_{a,i}'^o)U$. This means that $(k_{a,i}^o, k_{b,i}, t_{k,i})$ and $( k_{a,i}'^o, k_{b,i}'^o, t'_{k,i})$ breaks the binding property of $(\textsf{Setup}, \textsf{LinkTag})$. Another sub-case is that $( k_{a,i}'^o, k_{b,i}'^o, t'_{k,i})$ does \textit{not} satisfy proving relations involving those values, yet $\mathcal{A}$ can non-negligibly create accepted proofs, but this breaks the soundness property of ownership and unspentness proof and membership proof. Hence for all sub-cases, least one supposed property of a cryptographic construction is broken.

For the second case, assume that $b_0 \wedge b_2 = 1$. One sub-case is that the new e-notes are constructed honestly and amount balance $\sum_{i=1}^n{C_i'}=C_c + C_t$ (where $C_c$ is the ``change'' e-note) is satisfied, but $\sum_{i=1}^n{a_i} < a_t$. This implies that there exists $i\in\{1,\ldots,n\}$ such that $C'_i = v_{c,i} G + a_i H = v'_{c,i} G + a'_i H$ (where $v'_{c,i}\in\mathbb{F}$ is also computed by $\mathcal{A}$) \textit{and} $a'_i > a_i$. Thus $( v_{c,i}, a_i)$ and $(v'_{c,i}, a'_i)$ breaks the binding property of $(\textsf{Setup}, \textsf{PedersenC})$. Another sub-case is that $\mathcal{A}$ can non-negligibly create an accepted range proof for the amount $a$ committed in a new e-note such that $C \ne x G + a H$ or $\neg (0 \le a \le a_{max})$, but this breaks the soundness property of range proof. Hence for all sub-cases, least one supposed property of a cryptographic construction is broken.
\end{proof}
\begin{proof}[Proof of Theorem \ref{thm-prv}]
Assume that $\mathcal{A}$ succeeds in the $\textsf{PRV}$ experiment with non-negligible probability. There are three cases, each with sub-cases: the sender is distinguished, the receiver is distinguished, or the sent amount is distinguished.

For the first case, assume that $\mathcal{A}$ distinguished the sender. One sub-case is that . Hence for all sub-cases, least one supposed property of a cryptographic construction is broken.

For the second case, assume that $\mathcal{A}$ distinguished the receiver. One sub-case is that . Hence for all sub-cases, least one supposed property of a cryptographic construction is broken.

For the third case, assume that $\mathcal{A}$ distinguished the sent amount. One sub-case is that $\mathcal{A}$ determined that the $C$ of the new e-note for $\texttt{recv}_b$ hides $a_{\mathcal{A}, b}$, but this breaks the hiding property of $(\textsf{Setup}, \textsf{PedersenC})$. Another sub-case is that $\mathcal{A}$ recognized in $T$ that $C'_i - t_{c, i}G = x_{b,i}G + a_{b,i}H$ for some $i \in \{1, \ldots, n\}$, but this breaks the randomness of $t_c$. Another subcase is that the proving systems leak information related to amounts, but this breaks the perfect special HVZK of range proof and witness indistinguishability of membership proof. Moreover, leaking $t_c$ from membership proof leads to the previous case. The last subcase is that $\mathcal{A}$ can non-negligibly decrypt $\overline{a_{\mathcal{A},b}}$ without correct key, but this breaks the IND-CCA2 property of authenticated symmetric encryption scheme. Hence for all sub-cases, least one supposed property of a cryptographic construction is broken.
\end{proof}
\begin{proof}[Proof of Theorem \ref{thm-nsland}]
\end{proof}
\end{document}