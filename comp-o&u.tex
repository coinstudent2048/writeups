\documentclass{article}
\usepackage[top=1.0in,bottom=1.0in,left=1.0in,right=1.0in]{geometry}
\usepackage{amsmath,amssymb,amsthm,amsfonts}
\usepackage[utf8]{inputenc}
\usepackage{hyperref}

\newtheorem{definition}{Definition}[section]
\newtheorem{theorem}{Theorem}[section]
\newtheorem{lemma}[theorem]{Lemma}
\newtheorem*{remark}{Remark}

\title{Another Composition Ownership and Unspentness Proof for Seraphis}
\author{coinstudent2048}
\date{\today}

\begin{document}

\maketitle

\begin{abstract}
One component of Seraphis \cite{seraphis, seraphis2} is the ownership and unspentness proof. A composition proving system for Seraphis is presented that is simply an instantiation of the Modified Chaum-Pedersen Proving System presented in Appendix A of \cite{lelantus-spark}. Consequently, the proving system is complete, special sound, and special honest-verifier zero knowledge (SHVZK).
\end{abstract}

\section{Public parameters}
Let $\mathbb{G}$ be a prime order group where the Discrete Logarithm (DL) and Decisional Diffie-Hellman (DDH) problems are hard, and let $\mathbb{F}$ be its scalar field. Let $G, X, U$ be generators of $\mathbb{G}$ with unknown DL relationship to each other. Note that these generators may be produced using public randomness. Let $\mathcal{H}:\{0,1\}^*\rightarrow\mathbb{F}$ be a cryptographic hash function. We assume that $\mathcal{H}$ is a random oracle, hence we work in the random oracle model.

The notation $\xleftarrow{\$}$ will be used to denote for a uniformly randomly chosen element, and $(1/x)$ for the modular inverse of $x\in\mathbb{F}$. Lastly, additive notation is used for group operations.


\section{Composition Proving System}
The composition proving system is a protocol for the relation:
\begin{multline*}
\Big\{\big(G, X, U\in\mathbb{G}, \{K_i\}_{i=1}^n, \{\tilde{K}_i\}_{i=1}^n \in\mathbb{G}^n; \{x_i\}_{i=1}^n, \{y_i\}_{i=1}^n, \{z_i\}_{i=1}^n \in\mathbb{F}^n\big): \\ \bigwedge_{i=1}^n{\big(y_i \ne 0 \wedge K_i = x_i G + y_i X + z_i U \wedge \tilde{K}_i = (z_i/y_i)U\big)} \Big\}
\end{multline*}
Observe that if $n = 1$, then the relation reverts back to the proving relation shown in Subsection 3.1 of \cite{seraphis2}, with differences only in notation. In this composition proving system, the Prover only needs to produce one ownership and unspentness proof transcript for all $i$ instead of one proof transcript for each $i$.

The protocol proceeds as follows:
\begin{enumerate}
\item The prover generates $q\xleftarrow{\$}\mathbb{F}$ and $r_i, s_i \xleftarrow{\$}\mathbb{F}\ ,\ \forall i\in\{1,\ldots,n\}$. The prover computes
\begin{align*}
A_1 &= qG + \sum_{i=1}^n{r_i X} + \sum_{i=1}^n{s_i U} \\
A_{2,i} &= r_i \tilde{K}_i - s_i U\ ,\ \forall i\in\{1,\ldots,n\}
\end{align*}
and sends these values to the verifier.
\item The verifier sends a challenge $c\xleftarrow{\$}\mathbb{F}$ to the prover.
\item The prover computes the responses:
\begin{align*}
t_1 &= q + \sum_{i=1}^n c^i x_i \\
t_{2,i} &= r_i + c^i y_i\ ,\ \forall i\in\{1,\ldots,n\} \\
t_3 &= \sum_{i=1}^n (s_i + c^i z_i)
\end{align*}
and sends these values to the verifier.
\item The verifier checks the following equalities. If any of them fail, then the prover has failed to satisfy the composition proof system.
\begin{align*}
A_1 + \sum_{i=1}^{n}{c^i K_i} &= t_1 G + \sum_{i=1}^n{t_{2,i} X} + t_3 U \\
\sum_{i=1}^{n} A_{2,i} &= \sum_{i=1}^{n} t_{2,i}\tilde{K}_i - t_3 U \\
\end{align*}
\end{enumerate}
Using $\mathcal{H}$, it should be straightforward to apply Fiat-Shamir heuristic \cite{fiat-shamir} to the above protocol to make it non-interactive.

The above protocol is basically the Modified Chaum-Pedersen Proving System presented in Appendix A of \cite{lelantus-spark}, expect that the $U = x_i T_i + y_i G$ in the proving relation there becomes $0 = x_i T_i - y_i G$ here. Hence, the proof that the above protocol is complete, special sound, and SHVZK is essentially the same as in the original.

Lastly, to aid in cross-checking, the notation changes (indicated by $\rightarrow$) from the original in \cite{lelantus-spark} to here is shown below.
\begin{align*}
S_i &\rightarrow K_i&\ T_i &\rightarrow \tilde{K}_i \\
x_i &\rightarrow y_i&\ F &\rightarrow X \\
y_i &\rightarrow z_i&\ G &\rightarrow U \\
z_i &\rightarrow x_i&\ H &\rightarrow G
\end{align*}
\bibliographystyle{plain}
\bibliography{comp-o&u}
\end{document}